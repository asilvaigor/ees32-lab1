\section{Fundamentação Teórica}
Será apresentado o raciocínio que levou o grupo ao desenvolvimento de um algorítimo para a geração de códigos lineares. Para isso, serão apresentados alguns conceitos relativos a códigos lineares, bem como os principais problemas na geração de códigos.
 
\hfill 30 de Setembro de 2018.

\subsection{Códigos lineares e as matrizes G e H}
O conjunto $\textit{B} = \{0, 1\}$ com a operação de multiplicação usual e soma resto 2 é um corpo. Dessa modo, códigos das enuplas $(b_1, b_2 ...b_{n-1}, b_n)$, de números do corpo $\textit{B}$ definem um espaço vetorial. Pode-se, portanto, utilizar todas as propriedades de álgebra linear para tais códigos (código linear).

O princípio de correção de erros em um código linear é a introdução redundâncias na transmissão. As propriedades de álgebra linear ajudam a formular uma maneira inteligente de introduzir redundância nos códigos.

O método é o seguinte: supõe-se que se deseja transmitir um código de $k$ bits com $p$ redundâncias. Para isso, realiza-se uma transformação linear que leva um espaço de dimensão $k$ a um subespaço vetorial de dimensão $k$ num espaço de dimensão $p+k$. Isso pode ser feito por meio de uma transformação linear, introduz-se, logo, a matriz de transformação linear $\textbf{G}$.

Dessa maneira, para levar um vetor $\textbf{x}$ de informação a um vetor $\textbf{y}$ de código, basta fazer como em \ref{eq:encoding}:
\begin{equation}
    \textbf{y} = \textbf{x}\textbf{G}.
    \label{eq:encoding}
\end{equation}
Uma forma interessante para $\textbf{G}$ é a seguinte:
\begin{equation}
    \textbf{G} = [\textbf{I}_k \textbf{P}],
    \label{eq:G}
\end{equation}
em que $\textbf{I}_k$ é a identidade de ordem $k$. Nessa forma, os $k$ primeiros bits são exatamente os bits de informação (os $\textit{k}$ bits da enupla original).

Observa-se que a taxa do código, ou seja o número de bits de informação por bit transmitido, é $\frac{k}{n}$, com $n = p+k$. Por fim, sabe-se que todos os vetores $\textbf{y}$ estão em um subespaço vetorial de dimensão $k$ no espaço de dimensão $n$. Esse subespaço pode ser expresso, também, como núcleo de uma transformação linear (como mostrado em \ref{eq:kernel}) que leva o espaço $n$ a outro espaço.

\begin{equation}
	\textbf{y}\textbf{H}^T = \textbf{0}
	\label{eq:kernel}
\end{equation}

Do teorema do núcleo e da imagem a dimensão da imagem desta transformação deve ser $n-k = p$. Com isso, pode-se tomar esta imagem como sendo o espaço vetorial das enuplas de $p$ entradas. De \ref{eq:kernel} e \ref{eq:encoding},  $\textbf{x}\textbf{G}\textbf{H}^T =\textbf{0}$. Como isso vale para todo $\textbf{x}$, deve-se ter a relação \ref{eq:fundamental}:

\begin{equation}
	\textbf{G}\textbf{H}^T =\textbf{0}
	\label{eq:fundamental}
\end{equation}

Tomando $\textbf{H}$ da forma $\textbf{H} = [-\textbf{P}^T\textbf{I}_p]$, tem-se $\textbf{G}\textbf{H}^T = \textbf{0}$ (basta fazer a multiplicação das matrizes em blocos). Vale observar que $-\textbf{P}^T = \textbf{P}^T$.

\subsection{Princípio para detecção erros}

Para corrigir os erros é necessário um critério de correção. A presença de transmissão faz com que o valor recebido seja $\textbf{r} = \textbf{e}+\textbf{y}$, em que $\textbf{r}$ é o valor recebido e $\textbf{e}$ é um vetor de erros. 

Percebe-se que: 
\begin{equation}
    \textbf{r}\textbf{H}^T = \textbf{e}\textbf{H}^T,
    \label{eq:decoding}
\end{equation}

Uma vez que $\textbf{y}\textbf{H}^T = 0$. Dessa maneira, será associado a cada vetor de saída de $\textbf{r}\textbf{H}^T$ um tipo de erro diferente. Como a saída da transformação por $\textbf{H}^T$ tem dimensão $p$, há até $2^p$ possíveis erros detectáveis.

Nota-se que $2^p$ é um limite superior e, como será visto, dependendo da escolha da matriz $\textbf{H}^T$ será mais fácil detectar erros.

Defini-se como síndrome $\textbf{s}$, uma saída de $\textbf{r}\textbf{H}^T$. Logo há $2^p$ possíveis síndromes.

\subsection{Critério para qualidade do código}

O principal critério para a qualidade de um código em detecção de erros é a distância mínima do código. Para definir distância num código usa-se o peso de Hamming, o qual é simplesmente a soma do número de números 1 em um vetor do código.

Desse modo, a distância mínima é definida como em \ref{eq:dmin}:

\begin{equation}
	min(D(\textbf{v}_1-\textbf{v}_2))
	\label{eq:dmin}
\end{equation}

 Em que $D(\textbf{v})$ é o peso de Hamming do vetor $\textbf{v}$ e os vetores $\textbf{v}_1$ e $\textbf{v}_2$ são vetores quaisquer do código. Como o código é linear, pode-se escrever $\textbf{v}_3 = \textbf{v}_1-\textbf{v}_2$, também pertencente ao código. Assim, a distância mínima é o menor peso de Hamming do código.

Retornando à matriz $\textbf{H}^T$, supondo que há $t$ números 1 no vetor de erro, a síndrome correspondente a $\textbf{e}\textbf{H}^T$ se torna a combinação de $t$ linhas de $\textbf{H}^T$. Note que, se a combinação das $t$ linhas for $\textbf{0}$, este vetor pertence ao código. Dessa forma, o menor número de linhas de $\textbf{H}^T$ que se deve combinar para obter $\textbf{0}$ é equivalente à distância mínima.

\subsection{Distância mínima necessária}

Suponha que deseja-se detectar $u$ erros, logo, para cada um desses $u$ erros, deve haver uma síndrome distinta. Suponha agora, que há dois erros distintos $\textbf{e}_1$ e $\textbf{e}_2$, ambos produzindo a mesma síndrome. Podemos combinar as linhas da matriz $\textbf{H}^T$ utilizadas para produzir a síndrome por $\textbf{e}_1$ e por $\textbf{e}_2$ e isso produzirá o vetor $\textbf{0}$. Com isso, combinaram-se pelo menos $2u$ linhas para obter o vetor $\textbf{0}$.

Observa-se que se a distância mínima for $2u+1$ a hipótese acima nunca pode ocorrer, pois é preciso combinar pelo menos $2u+1$ linhas para obter $\textbf{0}$.

O problema agora é obter a melhor matriz $\textbf{H}^T$ para atender a esse critério (maior número mínimo de linhas que se deve combinar para obter $\textbf{0}$).

\subsection{Matriz escolhida}

A matriz $\textbf{P}^T$ é a parte de $\textbf{H}^T$ que se deve determinar. Como o código terá taxa $\frac{4}{7}$, tem-se as escolhas $[(p=6, k=8), (p=9, k=12), (p=15, k=20)...]$. O grupo escolheu determinar a matriz com o menor vetor de código com essa taxa que pode corrigir dois erros.

A escolha foi o código $(p=9, k=12)$, pois há mais síndromes do que possibilidades de erros duplos + erros simples, o que não ocorre para $(p=6, k=8)$.