\section{Algoritmo}

Os procedimentos desenvolvidos para a simulação e análise das codificações podem ser separados em dois tipos: geração da matriz $\textbf{H}^T$ e a efetuação dos cálculos para encodificação e decodificação do sinal.

\subsection{Geração de $\textbf{H}^T$}
Para a geração de matrizes $\textbf{H}^T$ de diferentes tamanhos desejados, foi desenvolvido um algoritmo guloso que, dado o número de colunas $p$ da matriz $\textbf{H}^T$ desejada e um valor mínimo de distância de Hamming admitida $d$, ele gera uma matriz de forma que sejam necessárias somar pelo menos $d$ linhas para obter o vetor nulo. 

A matriz a ser encontrada deve ter dimensão $\frac{7p}{3} \times n$, e as $p$ primeiras linhas compõem a matriz identidade. As outras $\frac{7p}{3} - p$ linhas são encontradas testando-se as permutações de vetores contendo de $d - 1$ a $p$ elementos de valor $1$. Note que, se fossem testados os vetores contendo menos de $d - 1$ valores $1$, seria possível somar linhas da matriz identidade com o vetor em questão e encontrar o vetor nulo, o que não satisfaz a condição admitida. Para cada vetor testado, se as combinações de até $d - 2$ outras linhas com o vetor não der o vetor nulo, este vetor é tomado como linha da matriz $\textbf{H}^T$.

O pseudocódigo está disposto no Algoritmo 1. Neste algoritmo, todas as combinações lineares de pelo menos $d - 2$ linhas de $\textbf{H}^T$ são armazenadas em $combinations$, de forma que combinações de uma linha são armazenadas em $combinations[0]$, combinações de duas linhas são armazenadas em $combinations[1]$, etc. Note que não é necessário armazenar as combinações de $d - 1$ linhas, somente garantir que nenhuma delas será o vetor nulo. Este vetor é essencial para garantir que as combinações não serão recalculadas a cada passo do algoritmo.

O vetor $tempComb$ também foi utilizado para evitar duplo processamento, armazenando temporiariamente as combinações lineares de forma que, se a combinação de no máximo $d - 1$ linhas nao resultar no vetor nulo, ele é concatenado no vetor $combinations$. 

\begin{algorithm}
\label{alg:geracao_matriz}
\caption{Geração gulosa da matriz $\textbf{H}^T$}
\begin{algorithmic}[1]
\State $H^T \gets 0$
\State $combinations \gets \emptyset_{d-2}$
\State $possibleRows \gets rows(I_n) \cup permutations(n, d - 1)$
\For {$row$ in  $possibleRows$}
    \State \textsc{InsertRow($row$)}
\EndFor

\Function{InsertRow($row$)}{}
    \State $tempComb \gets \emptyset_{d-2}$
    \State $tempComb[0] \gets row$
    \For {$c \in [0, d-3]$}
        \For {$v \in combinations[c]$}
            \If {$row \oplus v = 0$} 
                \State \textbf{end\ function}
            \EndIf
            \If {$c \neq d - 3$}
                \State $tempComb[c + 1] \gets tempComb[c+1] \cup (row \oplus v)$
            \EndIf
        \EndFor
    \EndFor
    \State $combinations \gets combinations \cup tempComb$
    \State $H^T \gets \begin{bmatrix} H^T \\ row \end{bmatrix}$
\EndFunction
\end{algorithmic}
\end{algorithm}

Neste código, primeiramente as variáveis são inicializadas. As possíveis linhas a serem adicionadas na matriz são armazenadas em $possibleRows$, inserindo-se as linhas da matriz identidade e todas as permutações de vetores de $n$ elementos contendo no mínimo $d - 1$ valores $1$. No próximo passo cada possibilidade de linha é testada para inserção na matrix $\textbf{H}^T$. Note que, como as primeiras linhas compõem a matriz identidade, ela é inserida na matriz primeiro. 

Para cada linha $row$ testada o vetor $tempComb$ é inicializado com a linha na sua primeira posição. Então, para cada $c \in [0, d-3]$, se as somas de elementos de $combinations[c]$ com $row$ for diferente do vetor nulo, estas somas são inseridas no vetor temporário $tempComb$ para posteriormente serem inseridas em $combinations$. Note que, se para qualquer valor de $c$ a soma testada resultar no vetor nulo, a linha $row$ e os vetores armazenados em $tempComb$ são descartados pois aquele vetor é indesejável.

\subsection{Codificação e decodificação}

Para fazer os códigos de codificação e decodificação dada uma matriz $\textbf{H}^T$, basta seguir as equações \ref{eq:fundamental}, \ref{eq:G}, \ref{eq:encoding} e \ref{eq:decoding}. Por meio de $\textbf{H}^T$ encontramos a matriz $\textbf{P}$ e consequentemente a matriz $\textbf{G}$. 

A codificação é encontrada simplesmente por multiplicação de matrizes. Para a decodificação, é necessário também gerar previamente um mapa que leva cada possível valor de síndrome obtido para um padrão de erro. Como um valor de síndrome pode ser resultado de vários padrões de erros diferentes, os erros de menos número de bits são priorizados no mapeamento, pois estes tem maior probabilidade de ocorrer. O Algoritmo 2 contém o pseudocódigo que encontra esse mapeamento.

\begin{algorithm}
\label{alg:mapeamento}
\caption{Mapeamento de síndromes em padrões de erros.}
\begin{algorithmic}[1]
\State $map \gets 0_{2^n \times \frac{7n}{3}}$
\For {$k \in [1, b]$}
    \State $errors \gets permutations(2^n, k)$
    \For {$error \in errors$}
        \State $sindromes \gets sindromes \cup (error \times \textbf{H}^T)$
    \EndFor
    \For {$s \in indexes(sindromes)$}
        \If {$sindromes[s] \notin map$}
            \State $map[sindrome[s]] \gets errors[s]$            
        \EndIf
    \EndFor
\EndFor
\State $map[0_{\frac{7n}{3}}] \gets 0_{\frac{7n}{3}}$
\end{algorithmic}
\end{algorithm}

Neste código, são gerados os possíveis vetores de erros em até $b$ bits e estes são multiplicados pela matriz $\textbf{H}^T$, obtendo-se as síndromes correspondentes. Então cada valor de síndrome obtida é associada ao erro que a gerou e armazenada na estrutura $map$. Para priorizar os erros em menos bits, se aquele valor de síndrome já foi mapeado, ele não é mapeado novamente. Por último o elemento do mapa que associa a síndrome vetor nulo é o vetor nulo (caso de não haver erro em nenhum bit). O valor de $b$ é escolhido arbitrariamente, de forma a tentar garantir que todos os valores de síndromes serão mapeados. Note que, caso alguma síndrome não tenha sido mapeada, ela será associada ao erro em nenhum bit, então é ideal que o valor de $b$ seja suficientemente grande. Para todos os experimentos realizados neste trabalho, o valor de $b$ adotado foi igual a $\frac{n}{3} + 1$.
